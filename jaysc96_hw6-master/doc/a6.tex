\documentclass{article}

\usepackage{fullpage}
\usepackage{color}
\usepackage{amsmath}
\usepackage{url}
\usepackage{verbatim}
\usepackage{graphicx}
\usepackage{parskip}
\usepackage{amssymb}
\usepackage{listings} % For displaying code

\begin{document}

% Colors
\definecolor{blu}{rgb}{0,0,1}
\def\blu#1{{\color{blu}#1}}
\definecolor{gre}{rgb}{0,.5,0}
\def\gre#1{{\color{gre}#1}}
\definecolor{red}{rgb}{1,0,0}
\def\red#1{{\color{red}#1}}
\def\norm#1{\|#1\|}
\newcommand{\argmin}[1]{\mathop{\hbox{argmin}}_{#1}}
\newcommand{\argmax}[1]{\mathop{\hbox{argmax}}_{#1}}
\def\R{\mathbb{R}}
\newcommand{\fig}[2]{\includegraphics[width=#1\textwidth]{#2}}
\newcommand{\centerfig}[2]{\begin{center}\includegraphics[width=#1\textwidth]{#2}\end{center}}
\def\items#1{\begin{itemize}#1\end{itemize}}
\def\enum#1{\begin{enumerate}#1\end{enumerate}}
\def\answer#1{\iftoggle{answers}{\blu{Answer}:\\#1}}
\newcommand{\code}[1]{\lstinputlisting[language=Matlab]{#1}}
\newcommand{\half}{\frac 1 2}
\def\argmax{\mathop{\rm arg\,max}}
\def\argmin{\mathop{\rm arg\,min}}
\def\rubric#1{\gre{Rubric: \{#1\}}}{}


\title{CPSC 340 Assignment 6 (due Friday April 7 at 11:59pm)}
\author{Neural Networks}
\date{}
\maketitle


\section*{Instructions}
\rubric{mechanics:3}

The above points are allocated for following the general homework instructions on GitHub.

\vspace{1em}

Note: this assignment is a bit more open-ended than the previous ones. There aren't really ``correct answers''. 
Try to have fun with it and learn something along the way. We'll mark generously... no need to go overboard!

\section{Neural Networks}
\rubric{reasoning:5}

If you run \texttt{python main.py -q 1} it will train a neural network on the \emph{basisData.pkl} data from Assignment~3. 
However, in its current form it uses no hidden layers and therefore is just performing linear regression.
In fact, if you take a look at the figure that is generated, you should see that it matches the figure
from Assignment~3 Question 2.1. 
Try to improve the performance of the neural network by adding hidden layers and tuning the other hyperparemeters. 
\blu{Hand in your plot after changing the code to have better performance, and list the changes you made. Write a couple sentences
explaining why you think your changes improved the performance. When appropriate, refer to concepts from the course like overfitting
or optimization.}

To see the list of hyperparameters and their definitions, see the scikit-learn MLPRegressor documentation:
\url{http://scikit-learn.org/stable/modules/generated/sklearn.neural_network.MLPRegressor.html}. Note: ``MLP'' stands for Multi-Layer Perceptron, 
which is another name for artificial neural network.\\
Also: we didn't discuss L-BFGS in class, but it may help to set \verb|solver| to \verb|lbfgs| since the data set is tiny.

\centerfig{.7}{../figs/modifiedBasisData.pdf}
I changed the Regressor to 100 neurons in the 1st hidden layer, and added an alpha of 0.01 to regularize the error, and optimize the hyper-parameters, which in turn reduces overfitting. Came to the number 100 after trying various numbers between 10-1000 and tried adding layers as well, but came to the conclusion only one hidden layer was present.


\section{Your Own Data}
\rubric{reasoning:5}

Try a neural network on the data set of your choosing. 
You can use a data set that's built into scikit-learn
(\url{http://scikit-learn.org/stable/datasets/}), or you can use a data set from a previous homework assignment, or something from an area that interests you. 
If you are doing classification rather than regression, use scikit-learn's MLPClassifier rather than the MLPRegressor that we've provided in the assignment code: \url{http://scikit-learn.org/stable/modules/generated/sklearn.neural_network.MLPClassifier.html}


\blu{Try fiddling with the hyperparameters. Write a few sentences about what you tried, what train/validation error you achieved,
and whether you think neural networks are a good model for your dataset.} Optionally (for glory rather than marks), install Keras/TensorFlow and compare it with
the scikit-learn neural net.

For the Iris dataset in scikit, on using the MLPClassifier with no hidden layers gives Training error $=  0.0$ and Validation error $=  0.6$, while increasing the number of layers doesn't really decrease the validation error, on repeated runs, with and without hidden layers give very similar errors.

\end{document}